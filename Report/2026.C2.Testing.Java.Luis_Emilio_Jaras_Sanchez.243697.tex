\documentclass[12pt]{article}
\usepackage[utf8]{inputenc}
\usepackage{graphicx}
\usepackage{geometry}
\usepackage{hyperref}
\usepackage{helvet}
\renewcommand{\familydefault}{\sfdefault} % Fuente Arial-like
\usepackage{fancyhdr}

\geometry{a4paper, margin=1in}

\begin{document}

% Portada
\begin{titlepage}
    \centering
    \vspace*{2cm}
    
    {\Huge\bfseries Reporte de Pruebas Automatizadas\par}
    \vspace{1cm}
    {\Huge\bfseries SauceDemo\par}
    
    \vspace{2cm}
    
    {\Large\bfseries Pruebas Funcionales E2E con\par}
    {\Large\bfseries Selenium (Java) y Cypress\par}
    
    \vfill
    
    {\Large\bfseries Alumno: Luis Emilio Jaras Sanchez\par}
    \vspace{0.5cm}
    {\large Matrícula: 243697\par}
    \vspace{0.5cm}
    {\large Curso: 2026.C2.Testing.Java\par}
    
    \vspace{1cm}
    {\large \today\par}
\end{titlepage}

% Índice
\tableofcontents
\newpage

\section{Introducción}

\begin{itemize}
    \item \textbf{Proyecto:} Pruebas E2E en SauceDemo (\url{https://www.saucedemo.com})
    \item \textbf{Herramientas Utilizadas:} Selenium (Java) y Cypress
    \item \textbf{Objetivo:} Iniciar sesión con los 6 usuarios de prueba proporcionados, agregar 4 artículos al carrito, validar el estado del carrito, generar logs en consola, y tomar capturas de pantalla de los resultados. Además, implementar el manejo de excepciones para los usuarios con errores intencionales de la plataforma.
\end{itemize}

\section{Desarrollo}

Se dividió el proyecto en dos enfoques principales y se organizó en carpetas separadas:

\subsection{Enfoque 1: Selenium con Java (\texttt{/Selenium-Java})}
Se creó un proyecto en Maven utilizando JUnit 5 y WebDriverManager para facilitar la configuración del Chrome Driver.

\textbf{Implementación (\texttt{SauceDemoTest.java}):}
Mediante un \texttt{@ParameterizedTest}, se itera sobre la lista de 6 usuarios.
\begin{itemize}
    \item \textbf{Acciones:} Login, espera explícita (\texttt{WebDriverWait}), agregar hasta 4 elementos y verificación del \texttt{shopping\_cart\_badge}.
    \item \textbf{Manejo de Errores:} Usuarios como \texttt{locked\_out\_user} fueron validados mediante comprobación del mensaje emergente de error. 
    \item \textbf{Capturas:} Al finalizar el escenario exitoso o la captura de una excepción, se toman capturas automatizadas y se guardan en la carpeta \texttt{/screenshots}.
\end{itemize}

\textbf{Comandos para ejecutar:}
\begin{enumerate}
    \item Navegar a la carpeta: \texttt{cd Selenium-Java}
    \item Ejecutar tests con Maven: \texttt{mvn test}
\end{enumerate}

\subsection{Enfoque 2: Cypress (\texttt{/Cypress-Testing})}
Se inicializó un proyecto de NPM y se instaló Cypress localmente.

\textbf{Implementación (\texttt{saucedemo.cy.js}):}
Para evitar que fallos del código interno de SauceDemo para el usuario \texttt{error\_user} hicieran fallar a la suite se agregó el interceptor \texttt{Cypress.on('uncaught:exception')}.
\begin{itemize}
    \item \textbf{Acciones:} \texttt{cy.visit}, login, iteración de los contenedores para buscar los botones de "Add to cart".
    \item \textbf{Manejo de Tiempos:} Cypress espera implícitamente a los elementos de UI, reduciendo fallos por latencia de carga.
\end{itemize}

\textbf{Comandos para ejecutar:}
\begin{enumerate}
    \item Navegar a la carpeta: \texttt{cd Cypress-Testing}
    \item Ejecutar Cypress: \texttt{npx cypress run}
\end{enumerate}

\section{Resultados y Capturas}
Los resultados de ejecución han guardado las capturas de pantalla para cada usuario. Como evidencia principal, se adjunta:

\begin{figure}[h]
    \centering
    % Ajusta esta ruta a donde alojes tus imágenes en Overleaf o localmente
    \includegraphics[width=0.8\textwidth]{../Selenium-Java/screenshots/standard_user_success.png}
    \caption{Evidencia: standard\_user exitoso (Selenium Java)}
\end{figure}

\begin{figure}[h]
    \centering
    % Ajusta esta ruta a donde alojes tus imágenes en Overleaf o localmente
    \includegraphics[width=0.8\textwidth]{../Selenium-Java/screenshots/locked_out_user_locked_out.png}
    \caption{Evidencia: locked\_out\_user error esperado (Selenium Java)}
\end{figure}

\section{Conclusiones}
La automatización de un mismo flujo E2E en dos herramientas diferentes permite contrastar enormemente sus enfoques:
\begin{enumerate}
    \item \textbf{Selenium Java (Imperativo):} Requiere configuración explícita del Driver, tiempos de espera y dependencias. Es altamente estricto, lo cual lo hace potente pero a la vez requiere mayor mantenimiento.
    \item \textbf{Cypress (Declarativo y Asíncrono):} Su configuración con JSON/JS es sumamente sencilla. Emula al usuario de una manera natural en la web moderna e inyecta esperas por defecto que reducen la "flakiness" de las pruebas. Resolver problemas técnicos de la propia página requirió solo interceptar el error a nivel del test runner sin ensuciar la lógica de negocio.
\end{enumerate}
Manejar estos escenarios fallidos es esencial para la estabilidad de cualquier ciclo de pruebas reales.

\vspace{1cm}
\textbf{Liga del Repositorio (Git):} \\
\url{https://github.com/EmilioJaras3/2026.C2.Testing.Java.Luis_Emilio_Jaras_Sanchez.243697.tex-en-la-carpeta-Report}

\end{document}
