\documentclass[12pt]{article}
\usepackage[utf8]{inputenc}
\usepackage{graphicx}
\usepackage{geometry}
\usepackage{hyperref}

\geometry{a4paper, margin=1in}

\title{\textbf{Reporte de Pruebas Automatizadas: SauceDemo}}
\author{
  Nombre: Luis Emilio Jaras Sanchez \\
  Matrícula: 243697 \\
  Curso: 2026.C2.Testing.Java
}
\date{\today}

\begin{document}

\maketitle

\section*{1. Portada}
\begin{itemize}
    \item \textbf{Proyecto:} Pruebas E2E en SauceDemo (\url{https://www.saucedemo.com})
    \item \textbf{Herramientas Utilizadas:} Selenium (Java) y Cypress
    \item \textbf{Objetivo:} Iniciar sesión con los 6 usuarios de prueba proporcionados, agregar 4 artículos al carrito, validar el estado del carrito, generar logs en consola, y tomar capturas de pantalla de los resultados. Además, implementar el manejo de excepciones para los usuarios con errores intencionales de la plataforma.
\end{itemize}

\section*{2. Desarrollo}

Se dividió el proyecto en dos enfoques principales y se organizó en carpetas separadas:

\subsection*{2.1 Enfoque 1: Selenium con Java (\texttt{/Selenium-Java})}
Se creó un proyecto en Maven utilizando JUnit 5 y WebDriverManager para facilitar la configuración del Chrome Driver.

\textbf{Configuración (\texttt{pom.xml}):}
Se agregaron dependencias para \texttt{selenium-java}, \texttt{webdrivermanager} y \texttt{junit-jupiter}. 

\textbf{Implementación (\texttt{SauceDemoTest.java}):}
Mediante un \texttt{@ParameterizedTest}, se itera sobre la lista de 6 usuarios.
\begin{itemize}
    \item \textbf{Acciones:} Login, espera explícita (\texttt{WebDriverWait}), agregar hasta 4 elementos y verificación del \texttt{shopping\_cart\_badge}.
    \item \textbf{Manejo de Errores:} Usuarios como \texttt{locked\_out\_user} fueron validados mediante comprobación del mensaje emergente de error. Para usuarios como \texttt{problem\_user} o \texttt{error\_user} (que interceptan los clics) se capturaron excepciones para no quebrar la ejecución de las demás pruebas.
    \item \textbf{Capturas:} Al finalizar el escenario exitoso o la captura de una excepción, se toman capturas automatizadas y se guardan en la carpeta \texttt{/screenshots}.
\end{itemize}

\textbf{Comandos para ejecutar:}
\begin{enumerate}
    \item Navegar a la carpeta: \texttt{cd Selenium-Java}
    \item Ejecutar tests con Maven: \texttt{mvn test} \textit{(Nota: Necesita Maven agregado al PATH)}
\end{enumerate}

\subsection*{2.2 Enfoque 2: Cypress (\texttt{/Cypress-Testing})}
Se inicializó un proyecto de NPM y se instaló Cypress localmente.

\textbf{Configuración (\texttt{cypress.config.js}):}
Se configuró el soporte para imprimir logs directamente en la terminal de Node mediante la función \texttt{on('task', \{ log(message)... \})}. Además, se añadió una regla a nivel de prueba (\texttt{Cypress.on('uncaught:exception')}) para evitar que fallos del código interno de SauceDemo para el usuario \texttt{error\_user} hicieran fallar a la suite.

\textbf{Implementación (\texttt{saucedemo.cy.js}):}
\begin{itemize}
    \item \textbf{Acciones:} \texttt{cy.visit}, login, iteración de los contenedores para buscar los botones de ``Add to cart''.
    \item \textbf{Manejo de Tiempos:} Cypress espera implícitamente a los elementos de UI, reduciendo fallos por latencia de carga.
\end{itemize}

\textbf{Comandos para ejecutar:}
\begin{enumerate}
    \item Navegar a la carpeta: \texttt{cd Cypress-Testing}
    \item Ejecutar Cypress: \texttt{npx cypress run}
\end{enumerate}

\section*{3. Conclusiones}
La automatización de un mismo flujo E2E en dos herramientas diferentes permite contrastar enormemente sus enfoques:
\begin{enumerate}
    \item \textbf{Selenium Java (Imperativo):} Requiere configuración explícita del Driver, tiempos de espera y dependencias (Mvn, JUnit). Es altamente estricto, lo cual lo hace potente pero a la vez requiere mayor mantenimiento y código repetitivo.
    \item \textbf{Cypress (Declarativo y Asíncrono):} Su configuración con JSON/JS es sumamente sencilla. Emula al usuario de una manera natural en la web moderna e inyecta esperas por defecto que reducen la ``flakiness'' de las pruebas. Resolver problemas técnicos de la propia página (como excepciones internas) requirió solo interceptar el error a nivel del test runner sin ensuciar la lógica de negocio.
\end{enumerate}
Manejar estos escenarios fallidos es esencial para la estabilidad de cualquier ciclo de pruebas reales.

\vspace{1cm}
\textbf{Liga del Repositorio (Git):} \\
[Incluir liga de GitHub aquí]

\end{document}
